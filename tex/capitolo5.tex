\chapter{Btlejuice and Gattacker}
\label{chapter5}
\thispagestyle{empty}

\noindent 
Man-In-The-Middle attacks are probably the most interesting cases on which to work, even though they are much harder to accomplish than what the theory may lead to think.

Basically the main objective of this attack, is to connect to the device, extract its informations and features, as well as the MAC address, and create a dummy device of it, copy of the original. Then wait for the connection and forward the messages to the destination.

In order to accomplish the attack, even if it may seem obvious, it's necessary to take into account the Bluetooth range and thus be near the device that has to be hacked.

There are different devices like Ubertooth or Adafruit Bluefruit sniffer that also provide MITM support, but they are quite expensive solutions.
However there two other tools that allow an hands on on BLE tampering, Btlejuice and Gattacker.

It is also important to note that, even if this is only a testing, the tools are real and potentially dangerous. As such it is always advised to be cautious and aware of what is doing.

\section{Btlejuice}

Btlejuice is a complete framework based on Node.js that includes an interception core, an interception proxy, a dedicated web interface and Python and Node.js bindings.
The requirement for this tool are the Bluetooth and Bluex suites and their respective libraries, as well as Node.js (>=4.3.2) and Npm.
While it may seem easy to install, Btejuice requires a lot of dependencies and may cause a few headaches to the user.
Through Npm then, it is possible to install Btlejuice.
A brief explanation is available on Btlejuice official page.


\begin{figure}
	\centering
	\includegraphics[caption= {Host Pov - Establishing the Connection with the proxy} ][width=0.9\textwidth]{host1}
	\label{fig:images\host1}
\end{figure}

\begin{figure}
	\centering
	\includegraphics[caption= {Proxy POV - Establishing the Connection with the host and copying the services of the device}][width=0.9\textwidth]{proxy1.png}
	\label{fig:images\proxy1}
\end{figure}

\begin{figure}
	\centering
	\includegraphics[caption= {Host Pov - Btlejuice UI} ][width=0.9\textwidth]{host2}
	\label{fig:images\host2}
\end{figure}

In order to use Btlejuice it is important to install Noble on the host machine and Bleno on the proxy, the one that will serve as dummy.
Both Noble and Bleno are Node.js modules to implement BLE peripherals.
Another relevant requirement is to have BT 4.0+ support, otherwise the machines won't detect any traffic.

Up to the current version, Btlejuice is able to provide support for Bluez 5.x and JustWorks BLE devices, others devices may not work.

Btlejuice main freatures are the live GATT operations and data sniffing; it also allows data manipulation on-the-fly and creating a copy of the device while being connected to it, using CSR adapters.
There are however some limitations: Noble doesn't support long writes and there is some latency due to the connection from BLE to websocket and then back to BLE. It is also very tricky to use (or may not work at all) when device keep connections or advertise during a short delay.

Btlejuice is very promising and has indeed some strong feature, but there is definitely the need for more testing and debugging.

\section{Gattacker}
Gattacker is another interesting instrument for MITM attacks, even though it is a little more complex than Btlejuice.
The installation is quite similar as it requires node and npm, as well as Bluetooth and bluez dependencies as well.

The deployment concept is basically the same: two machines, one host and one proxy, the first with Noble and the other with Bleno. 
The last step is, of course, to install Gattacker.

It is necessary to repeat the same procedure on both machines, and edit the configuration file in order to connect to the IP address of the slave (proxy) machine.

Then the two platforms run in parallel in order to clone the device and create a dummy.
It is a bit tricky to perform these passages as there is a sequence of steps to follow to achieved the desired result.
After saving the services and characteristics of the device to be copied, it is also possible to modify it, and then advertise using the hacked version of it.

Even Gattacker has a support to replay actions, they can be stored in log files and then replayed according to the user wishes, even from the BLESniffer mobile app.
Gattacker is definetely less intuitive to use than Btlejuice, and in this case too some more testing is definitely needed.