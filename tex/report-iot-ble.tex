\documentclass[11pt,a4paper,openright,twoside]{report}

\usepackage[pdftex]{graphicx}
\usepackage[english]{babel}
\usepackage[T1]{fontenc}
\usepackage[utf8]{inputenc}
\usepackage{url}
\usepackage[hidelinks]{hyperref}
\usepackage{setspace}
\usepackage{mwe}
\usepackage{pdfpages}

\usepackage{fancyhdr}
\usepackage{indentfirst}
\usepackage{newlfont}
\usepackage{xcolor}

\definecolor{mygreen}{rgb}{0,0.6,0}
\definecolor{mygray}{rgb}{0.5,0.5,0.5}
\definecolor{mymauve}{rgb}{0.58,0,0.82}
\definecolor{pred}{rgb}{0.9,0,0}
\definecolor{light-gray}{gray}{0.95}
\usepackage{listings}
\lstset{ 
	backgroundcolor=\color{light-gray},   % choose the background color; you must add \usepackage{color} or \usepackage{xcolor}; should come as last argument
	breakatwhitespace=false,         % sets if automatic breaks should only happen at whitespace
	breaklines=true,                 % sets automatic line breaking
	escapeinside={\%*}{*)},          % if you want to add LaTeX within your code
	extendedchars=true,              % lets you use non-ASCII characters; for 8-bits encodings only, does not work with UTF-8
	keepspaces=true,                 % keeps spaces in text, useful for keeping indentation of code (possibly needs columns=flexible)
	basicstyle=\footnotesize,        % the size of the fonts that are used for the code
	numbers=left,                    % where to put the line-numbers; possible values are (none, left, right)
	numbersep=5pt,                   % how far the line-numbers are from the code
	%language=XML,                	 % the language of the code
	commentstyle=\color{mygreen},    % comment style
	keywordstyle=\color{blue},       % keyword style
	numberstyle=\tiny\color{mygray}, % the style that is used for the line-numbers
	%stringstyle=\color{pred},
	rulecolor=\color{black},         % if not set, the frame-color may be changed on line-breaks within not-black text (e.g. comments (green here))
	stringstyle=\color{mymauve},     % string literal style
	tabsize=4		%modify tab space, default too wide
}

\usepackage{amssymb}
\usepackage{amsmath}
\usepackage{latexsym}
\usepackage{amsthm}

\oddsidemargin=30pt \evensidemargin=20pt%impostano i margini
\usepackage[htt]{hyphenat} %per andare a capo nei typescript
\hyphenation{uni-ver-sity
	uni-ver-sit-ies
	how-ever
	ma-nu-script
	ma-nu-scripts
	re-ci-pro-city
	through-out
	some-thing
	al-tern-ate}%serve per la sillabazione: tra parentesi 

\pagestyle{fancy}\addtolength{\headwidth}{20pt}
\renewcommand{\chaptermark}[1]{\markboth{\thechapter.\ #1}{}}
\renewcommand{\sectionmark}[1]{\markright{\thesection \ #1}{}}
\rhead[\fancyplain{}{\bfseries\leftmark}]{\fancyplain{}{\bfseries\thepage}}
\cfoot{}

%\linespread{1.3}                     

\begin{document}
	
	\includepdf{title-by-itself}
	\clearpage{\pagestyle{empty}\cleardoublepage}%non numera l'ultima pagina sinistra
	
	\include{indice}                        
	\clearpage{\pagestyle{empty}\cleardoublepage}
	\pagenumbering{arabic}                  %mette i numeri arabi

	\chapter{Introduction}
\label{Intro}
\thispagestyle{empty}

\noindent Bluetooth Low Energy devices are flooding the market of cheap, wearable, health, home-appliance electronics. Some examples include monitoring devices for sport, health and outdoor activities in general; they also connect many household appliances as well as security cameras and locks. Other interesting use-cases include wristbands used in theme parks or live music concerts (although the technology may differ).

Intrinsically, these devices perform very simple tasks, mostly in the field of data gathering. It seems totally harmless to have a smart watch tracking your position at any time of day, but what if it were possible for anyone to query it and extract its data? Their minimalism is what makes them so appealing to the general public, but it is sometimes their intrinsic weakness.

Moving a bit further from \textit{"hackers"} and information security, data gathering is a fundamental step for data analysis in a Big Data context. Privacy concerns should be taken into consideration also in terms of data anonymization.

In this project report, we will start by describing general device characteristics and Bluetooth specifications. We will then move to their concrete observation with hands-on analyses, taking into consideration different developing tools and devices. In particular, we will focus our attention on security risks and concrete flaws, highlighting security exploits when possible. Finally, we will summarize our findings, including a few ideas for future directions and developments.

For our tests, we used a Mi Band 2 smart band, a Magic Blue smart lightbulb and a ST Microelectronics IoT node (model \texttt{B-L475E-IOT1A2 Discovery kit}). The code was run both on Ubuntu 16.04 and Kali Linux 4.13. It is worth mentioning that we also tried to use a Ubuntu GNOME OS, virtualized via VMWare on Windows 10, but this option was soon discarded as VMWare does not provide the drivers for Bluetooth Low Energy. Another requirement is that Bluetooth adapters on the smartphone or laptop must support Bluetooth version 4.0 or higher, as it includes for the first time the LE specification.

The software at our disposal included the builtin Linux commands, as well as Wireshark and Android apps (namely BLEScanner and nRFConnect). As we will see, we also tested some open-source tools available from Github: bleah, btlejuice and gattacker.
	% !TeX spellcheck = en_GB
\chapter{Bluetooth Low Energy Specification}
\label{chapter2}
\thispagestyle{empty}

\noindent In this chapter we briefly report the main technical characteristics of the Bluetooth Low Energy Personal-Area-Network technology. Our main reference is the Core Specification 5.0 available on the official website.

\section{Network Topology and Devices}
Devices can be divided into three categories: Advertisers, Scanners and Initiators. 
Communication has to be started by an initiator device following an \textit{advertisement connectable packet} (\texttt{ADV\_IND}). Advertisement is performed on the primary channels, while bidirectional communication happens on one of the 37 secondary channels, decided during the connection procedure. When a connection is established, the initiator and the advertiser respectively become the master and slave devices; while the former can be involved in more than one communication, the slave device can belong to a single \textit{piconet} at a time. In fact, it is not possible to have a BLE device paired with more than one "master" device at the same time.

As we had the possibility to concretely program a BLE device, we show in Listing \ref{list:adv-conn} an extract of code in which the device is advertising \textit{connectable} packets.
\lstinputlisting[caption={Set up flags for device mode of operation: connectable},label={list:adv-conn},language=c++]{example-adv-conn.cpp}

Advertising may also be \textit{non-connectable} (\texttt{ADV\_NONCONN\_IND}): the LE device periodically sends its data on the main channel for every scanner to read. This is really convenient in terms of implementation, but it also represent the least secure solution, as data is sent in clear text. In Listing \ref{list:adv-nonconn} we highlight the different parameter used to set \textit{non-connectable} advertising.
\lstinputlisting[caption={Set up flags for device mode of operation: non connectable},label={list:adv-nonconn},language=c++]{example-adv-nonconn.cpp}

On the whole, the Specification defines six different advertising packets, among which we also report the request for additional information from the advertisement (\texttt{SCAN\_REQ}), it having a direct corresponding bash command.

\section{Decypher Advertising Packets}
Before considering Man-In-The-Middle techniques, we show how it is possible to intercept data from advertising devices, mentioned in the previous section.

For this experiment, we used the STM IoT node and tested it on both our Linux distributions. The board retrieves the temperature and humidity levels in the environment and broadcasts them as a non-connectable advertising packet, as we can see in Listing \ref{list:adv-temp-hum}.
\lstinputlisting[caption={Retrieve sensor values and update packet},label={list:adv-temp-hum},language=c++]{example-temphum.cpp}

Packets broadcasted by the board can be retrieved with any of the tools described at the end of the previous chapter. We analysed the payload and found that it follows this structure:
\begin{center}
	\textit{length - meaning - content}.
\end{center}
The length is expressed in bytes and it refers to the overall size of \textit{meaning} and \textit{content}. The former two is a label which has to be matched with the corresponding Bluetooth GAP Assigned Number (specification available on the website) to understand its meaning. Common examples are \texttt{0x09} for "Complete Local Name" or \texttt{0x0a} for "Power Level". In our case, the label corresponds to "Service data 16-bit UUID", which means that we should look at the next two bytes to have a more precise indication of the received data. Vendors usually publish a web page in which they list all their codes with the respective meaning, while in this example we can see that its value is added to the payload in \texttt{service\_data[0]}, and corresponds to \texttt{00 aa}.

Finally, the next two bytes encode the temperature and humidity in the room.

	\clearpage{\pagestyle{empty}\cleardoublepage}
	\addcontentsline{toc}{chapter}{Bibliography}
	\bibliographystyle{plainurl}
	\bibliography{bibl_tesi}
	
	\appendix
	
	\pagestyle{fancy} 
	\fancyfoot{}                                               
	\renewcommand{\chaptermark}[1]{\markboth{\appendixname\ \thechapter.\ #1}{}} 
	\renewcommand{\sectionmark}[1]{\markright{\thesection.\ #1}}         
	\fancyhead[LE,RO]{\bfseries\thepage}    
	
	\fancyhead[RE]{\bfseries\leftmark}    
	\fancyhead[LO]{\bfseries\rightmark}     
	\renewcommand{\headrulewidth}{0.3pt} 
	
	\chapter{GATT Assigned Numbers}
\label{appendixA}
\thispagestyle{empty}

\begin{figure}
	\centering
	\begin{minipage}{0.9\textwidth}
		\centering
		\includegraphics[width=0.8\textwidth]{GAP-assigned-numbers.png}
	\end{minipage}
	\label{fig1-GAP-numbers}
\end{figure}

\end{document}
