\chapter{More Security Consideration}
\label{chapter4}
\thispagestyle{empty}

\noindent 

Low energy devices are, namely, devices that have low energy consumption as their strong point; they support wireless technology and have an enormous range of applications.
However manufacturers often prioritize the life-expectancy and energy consumption of the devices instead of their protection from external attacks.
As we already explained how pairing can be achieved depending on the devices, and as thus provide a first layer of protection, in this chapter a number of threats and will be analysed. In addition, a brief explanation on the security layer in BLE architecture is provided.

\section{BLE Threats}
\subsection{Passive Eavesdropping}
As our experiments highlighted, it is extremely easy to intercept a connection from a BLE device, as there is no protection from external users that try to connect to it. Sometimes it may also be possible to just sniff data from a device without even being connected, and of course with no consent from the device in question.
Of course a certain level of security comes from the pairing itself. 
If the device is connected to a client (may be a trusted one), then there is no way another device can connect, even thought it may be possible for the attacker to eavesdrop during the connection phase.
In this case, smart phones' apps or even the tools analysed in the previous chapter such as Gatttool or Bluetoothctl, have no means of eavesdropping the exchanged data.
However it may be possible with more sophisticated or ad hoc instruments like Ubertooth or Nordic Semiconductor Dongle, made specifically for the purpose of intercepting packets.
Another option to guarantee a superior level of security is using encrypted data in a way that ill-intentioned listeners are not able to decipher the messages sent.

\subsection{Man-In-The-Middle}
In this case the attacker first listens to the communication, and then intercepts the messages sent from one device to another. Before being delivered to the destination the message is modified.
This is also called \textit{Active eavesdropping} due to the active role of the attacker, that remains invisible from the communicating devices. 
Even using a public key cryptography may not suffice against this kind of attack: if the attacker listens to the shared public key during the initial pairing phase and shares its own public key, the attack will be successful.
He may also use the original public key to encrypt messages to the original sender, in order to keep its presence hidden.
In this context there are tools that allow users to try hands on MITM attacks, like Btlejuice or Gattacker, that will be analysed more deeply in the next chapter.

\subsection{Identity Tracking}
BLE devices are often designed just to advertise data in a periodic way, updating their status or characteristics. However the packets contain the MAC address of the broadcaster and even information about the proximity of the device in terms of signal strength.
Using this information, and depending on the type of information the device advertise, it may be possible for the attacker to track the device, even more so if the data sent is unique and specific for a certain entity (either a person or a device).

\subsection{Duplicate Device}
Man-In-The-Middle frameworks such as Btlejuice or Gattacker are also an example of another threat that may befall BLE devices: the creation of dummy devices. Once the MAC address is obtained, all the services and the characteristics are copied and then the dummy device is activated.
This can be extremely dangerous.

\section{Architecture}
BLE devices implement the key management and security manager on host instead of controller and all the key generation and distribution is a responsibility of the Security Manager on host.
This approach is introduced by the Bluetooth Specifications and helps the host to be flexible and reduces the cost and complexity of the controller.
The SM defined the authentication, the pairing and encryption between the BLE devices and it uses the services provided by the L2CAP layer to manage all these functions.

\begin{figure}
	\centering	
	\includegraphics{width=0.5\textwidth}{architecture.gif}
	\label{fig:architecture}
\end{figure}

Security for BLE devices is expressed in LE Security Mode and may vary in a range from 1 to 4.
Each service and device may have a different security requirement.
\begin{itemize}
	\item LE Security Level 1: No secure communication and no pairing needed.
	\item LE Security Level 2: Support to authenticated and unauthenticated pairing but mandatory data signing with various techniques including public key cryptography.
	\item LE Security Level 3,4: More security to the system than the other levels.
	\item Mixed Security Mode: There may be devices that need Level 1 and 2 and thus may use a combination of different security modes depending on the need.
	\item Secure Connection Only: Authenticated connection and pairing with encryption. Devices accept only outgoing and incoming connections for services that use security mode Level 4.
\end{itemize}

In general however BLE devices are usually weakly protected: very few devices use encryption but they do not require it (even the failed pairing is fine), and almost all devices are not strongly authenticated by mobile applications.
In addition to this BD address is often the only check needed to ensure authenticity. On the other side, as we will explain in the next chapter, attacks like sniffing and MITM are more difficult to accomplish than it seems.


